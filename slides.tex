% DojoCon Presentation: Documenting Scratch with LaTeX
% Author: Benoit de Biolley
% Compile with: pdflatex

\documentclass{beamer}

% Theme
\usetheme{Madrid}

% Packages
\usepackage{scratch3}
\usepackage[utf8]{inputenc}
\usepackage{hyperref}

% Title Information
\title[Scratch + LaTeX]{Documenting Scratch Projects with LaTeX}
\subtitle{How to use the scratch3 package}
\author{Benoit de Biolley}
\date{DojoCon}

\begin{document}

% ---------------------------------------------------------
\begin{frame}
  \titlepage
\end{frame}
% ---------------------------------------------------------

\begin{frame}{Agenda}
  \tableofcontents
\end{frame}

% ---------------------------------------------------------
\section{Introduction}

\begin{frame}{Why Documentation?}
  \begin{itemize}
    \item Improves learning
    \item Helps others understand your project
    \item Encourages good coding habits
    \item Makes sharing easier
  \end{itemize}
\end{frame}

% ---------------------------------------------------------
\section{What is LaTeX?}

\begin{frame}{What is LaTeX?}
  \begin{itemize}
    \item Typesetting system for high-quality documents
    \item Widely used in education and research
    \item Ideal for reproducible project documentation
  \end{itemize}
\end{frame}

\begin{frame}[fragile]{Minimal Example}
  \begin{verbatim}
\documentclass{article}
\begin{document}
Hello DojoCon!
\end{document}
  \end{verbatim}
\end{frame}

% ---------------------------------------------------------
\section{The scratch Package}

\begin{frame}{scratch Package}
  \begin{itemize}
    \item Reproduces visual Scratch blocks in LaTeX
    \item Perfect for tutorials and class materials
    \item Simple syntax to represent scripts
  \end{itemize}
\end{frame}

\begin{frame}[fragile]{Simple Block}
  \begin{verbatim}
\begin{scratch}
  \blockmove{move \ovalnum{10}}
  \blockmove{turn \turnright{} \ovalnum{90} degrees}
\end{scratch}
  \end{verbatim}
\end{frame}

\begin{frame}{Rendered Example}
\begin{scratch}
  \blockmove{move \ovalnum{10}}
  \blockmove{turn \turnright{} \ovalnum{90} degrees}
\end{scratch}
\end{frame}

% ---------------------------------------------------------
\section{Documenting a Script}

\begin{frame}[fragile]{Example Script}
\vbox{
  \begin{verbatim}
\begin{scratch}
  \blockinit{when \greenflag clicked}
  \blockinfloop{forever}
  {
    \blockmove{go to \ovalmove*{mouse-pointer}}
    \blockmove{move \ovalnum{10} steps}
  }
\end{scratch}
  \end{verbatim}
}
\end{frame}



\begin{frame}{Rendered Script}
\begin{scratch}
  \blockinit{when \greenflag clicked}
  \blockinfloop{forever}
  {
    \blockmove{go to \ovalmove*{mouse-pointer}}
    \blockmove{move \ovalnum{10} steps}
  }
\end{scratch}
\end{frame}

\begin{frame}{\textbackslash selectmenu vs \textbackslash oval(suffix)}
  \setscratch{scale=.7}
  \begin{scratch}
    \blockinit{when \greenflag clicked \selectmenu{selectmenu}}
    \blockmove{bloc de mouvement \selectmenu{selectmenu} \ovalmove{ovalmove} \ovalmove*{ovalmove*}}
    \blocklook{bloc d’apparence \selectmenu{selectmenu} \ovallook{ovallook} \ovallook*{ovallook*}}
    \blocksound{bloc de son \selectmenu{selectmenu} \ovalsound{ovalsound} \ovalsound*{ovalsound*}}
    \blockpen{bloc de stylo \selectmenu{selectmenu} \ovalpen{ovalpen} \ovalpen*{ovalpen*}}
    \blockvariable{bloc de variable \selectmenu{selectmenu} \ovalvariable{ovalvariable}}
    \blocklist{bloc de liste \selectmenu{selectmenu} \ovallist{ovallist}}
    \blockevent{bloc d’événement \selectmenu{selectmenu} }
    \blockcontrol{bloc de contrôle \selectmenu{selectmenu} \ovalcontrol{ovalcontrol} \ovalcontrol*{ovalcontrol*}}
    \blocksensing{bloc de capteur \selectmenu{selectmenu} \ovalsensing{ovalsensing} \ovalsensing*{ovalsensing*}}
    \blockstop{stop \selectmenu{selectmenu}}
  \end{scratch}
  \end{frame}


% ---------------------------------------------------------
\section{Why LaTeX?}

\begin{frame}{Why Use LaTeX for Scratch?}
  \begin{itemize}
    \item Consistent formatting
    \item Professional output
    \item Easy version control (Git)
    \item Reusable templates
  \end{itemize}
\end{frame}

% ---------------------------------------------------------
\section{Workflow}

\begin{frame}{Typical Workflow}
  \begin{enumerate}
    \item Write .tex file
    \item Compile to PDF
    \item Share or publish
  \end{enumerate}
\end{frame}

% ---------------------------------------------------------
\section{Conclusion}

\begin{frame}{Conclusion}
  \begin{itemize}
    \item scratch + LaTeX = great for education
    \item Helps document, teach, and share
    \item Encourages clean project structure
  \end{itemize}
\end{frame}

\begin{frame}{Questions?}
  Thank you!
\end{frame}

\end{document}
