% DojoCon Presentation: Documenting Scratch with LaTeX
% Author: Benoît de Biolley
% Compile with: pdflatex

\documentclass{beamer}

% Theme
\usetheme{Madrid}

% Packages
\usepackage{scratch3}
\usepackage[utf8]{inputenc}
\usepackage{hyperref}

% Title Information
\title[Scratch + LaTeX]{Documenting Scratch Projects with LaTeX}
\subtitle{How to use the scratch3 package}
\author{Benoît de Biolley}
\date{DojoCon}

\begin{document}

% ---------------------------------------------------------
\begin{frame}
  \titlepage
\end{frame}
% ---------------------------------------------------------

\begin{frame}{Agenda}
  \tableofcontents
\end{frame}

% ---------------------------------------------------------
\section{Introduction}

\begin{frame}{Who am I ?}
  \begin{itemize}
    \item Coderdojo Sint-Pieters-Leeuw
    \item Coderdojo Forest
    \begin{figure}
        \centering
        \includegraphics[width=0.5\linewidth]{map.png}
        \caption{Coderdojo's in Brussels}
    \end{figure}
  \end{itemize}
\end{frame}
\begin{frame}{My dojo}
\begin{figure}
\begin{minipage}{.5\textwidth}
    \centering
    \includegraphics[width=0.95\linewidth]{cdjForest.jpg}
    \caption{Oktober}
\end{minipage}%
\begin{minipage}{.5\textwidth}
    \centering
    \includegraphics[width=0.95\linewidth]{cdjForest2.jpg}
    \caption{November}
\end{minipage}%
\end{figure}
\end{frame}

% ---------------------------------------------------------
\section{What is \LaTeX ?}

\begin{frame}{What is \LaTeX?}
\centering

\LaTeX
\vspace{5mm}
  \begin{itemize}
    \item Typesetting system for high-quality documents
    \item Widely used in education and research
    \item Ideal for reproducible project documentation
  \end{itemize}
\end{frame}

\begin{frame}[fragile]{Minimal Example}
\begin{minipage}{.49\textwidth}
  \begin{verbatim}
\documentclass{article}
\begin{document}
Hello DojoCon!
\end{document}
  \end{verbatim}
  \end{minipage}
\begin{minipage}{.49\textwidth}
Hello DojoCon!
\end{minipage}
\end{frame}

% ---------------------------------------------------------
\section{The scratch Package}

\begin{frame}{scratch Package}
  \begin{itemize}
    \item Reproduces visual Scratch blocks in \LaTeX
    \item Perfect for tutorials and class materials
    \item Simple syntax to represent scripts
  \end{itemize}
\end{frame}

\begin{frame}[fragile]{Simple Block}
  \begin{verbatim}
\begin{scratch}
  \blockmove{move \ovalnum{10} steps}
  \blockmove{turn \turnright{} \ovalnum{90} degrees}
\end{scratch}
  \end{verbatim}
\begin{scratch}
  \blockmove{move \ovalnum{10} steps}
  \blockmove{turn \turnright{} \ovalnum{90} degrees}
\end{scratch}
\end{frame}

% ---------------------------------------------------------
\section{Documenting a Script}

\begin{frame}[fragile]{Example Script}
\vbox{
  \begin{verbatim}
\begin{scratch}
  \blockinit{when \greenflag clicked}
  \blockinfloop{forever}
  {
    \blockmove{go to \ovalmove*{mouse-pointer}}
    \blockmove{move \ovalnum{10} steps}
  }
\end{scratch}
  \end{verbatim}
}
\end{frame}



\begin{frame}{Rendered Script}
\begin{scratch}
  \blockinit{when \greenflag clicked}
  \blockinfloop{forever}
  {
    \blockmove{go to \ovalmove*{mouse-pointer}}
    \blockmove{move \ovalnum{10} steps}
  }
\end{scratch}
\end{frame}

\begin{frame}{\textbackslash selectmenu vs \textbackslash oval(suffix)}
  \begin{minipage}{.65\textwidth}
      
  \setscratch{scale=.7}
  \begin{scratch}
    \blockinit{when \greenflag clicked \selectmenu{selectmenu}}
    \blockmove{bloc de mouvement \selectmenu{selectmenu} \ovalmove{ovalmove} \ovalmove*{ovalmove*}}
    \blocklook{bloc d’apparence \selectmenu{selectmenu} \ovallook{ovallook} \ovallook*{ovallook*}}
    \blocksound{bloc de son \selectmenu{selectmenu} \ovalsound{ovalsound} \ovalsound*{ovalsound*}}
    \blockpen{bloc de stylo \selectmenu{selectmenu} \ovalpen{ovalpen} \ovalpen*{ovalpen*}}
    \blockvariable{bloc de variable \selectmenu{selectmenu} \ovalvariable{ovalvariable}}
    \blocklist{bloc de liste \selectmenu{selectmenu} \ovallist{ovallist}}
    \blockevent{bloc d’événement \selectmenu{selectmenu} }
    \blockcontrol{bloc de contrôle \selectmenu{selectmenu} \ovalcontrol{ovalcontrol} \ovalcontrol*{ovalcontrol*}}
    \blocksensing{bloc de capteur \selectmenu{selectmenu} \ovalsensing{ovalsensing} \ovalsensing*{ovalsensing*}}
    \blockstop{stop \selectmenu{selectmenu}}
  \end{scratch}
  \end{minipage}
  \begin{minipage}{}
      \mbox{}\vspace*{0.2 cm}\textbackslash blockinit
      \vspace*{0.25 cm} \textbackslash blockmove
      \vspace*{0.25 cm} \textbackslash blocklook
      \vspace*{0.25 cm} \textbackslash blocksound
      \vspace*{0.25 cm} \textbackslash blockpen
      \vspace*{0.25 cm} \textbackslash blockvariable
      \vspace*{0.25 cm} \textbackslash blocklist
      \vspace*{0.25 cm} \textbackslash blockevent
      \vspace*{0.25 cm} \textbackslash blockcontrol
      \vspace*{0.25 cm} \textbackslash blocksensing
      \vspace*{0.25 cm} \textbackslash blockstop
  \end{minipage}
  \end{frame}


% ---------------------------------------------------------
\section{Why LaTeX?}

\begin{frame}{Why Use LaTeX for Scratch?}
  \begin{itemize}
    \item Consistent formatting
    \item Professional output
    \item Easy version control (Git)
    \item Reusable templates
  \end{itemize}
\end{frame}

% ---------------------------------------------------------
\section{Workflow}

\begin{frame}{Typical Workflow}
  \begin{enumerate}
    \item Write .tex file
    \item Compile to PDF
    \item Share or publish
  \end{enumerate}
\end{frame}

% ---------------------------------------------------------
\section{Conclusion}

\begin{frame}{Conclusion}
  \begin{itemize}
    \item scratch + LaTeX = great for education
    \item Helps document, teach, and share
    \item Encourages clean project structure
  \end{itemize}
\end{frame}

\begin{frame}{Questions?}
  Let's go !
  \newline
  \href{https://www.overleaf.com/}{https://www.overleaf.com/}
  \newline
  \href{https://ctan.org/pkg/scratch3}{https://ctan.org/pkg/scratch3}
  
\end{frame}

\end{document}
